Dieses abschließende Kapitel fasst die vorgestellten Themen und deren roten Faden zusammen und gibt einen Ausblick auf Arbeit die in den Forschungsgebieten noch erledigt werden muss und einen Abriss über Forschungsgebiete die nur gestreift werden konnten.

\section{Zusammenfassung}
Diese Arbeit beschäftgit sich mit den wichtigsten Themen die in \cite{misc:Mikkola2013} definiert wurden:

\begin{itemize}
	\item Sensorsysteme (GIS- und Sensornetzwerksysteme)
	\item Modellierung von für die Energieeffizienz in der Landwirtschaft relevanten Daten für den Austausch, Verarbeitung und Optimierung.
	\item Planungs- und Unterstützungssysteme 
\end{itemize}

Die Arbeiten haben gezeigt, dass Sensornetzwerksysteme sowohl in kontrollierten Umgebungen wie Glaushäusern eingesetzt werden können sowie auf großen freiliegenden Flächen. Die verfügbaren Protokolle sind soweit Auswertungen auf einzelne Knoten verlagern zu können. Für das Routing gibt es mehrere Lösungen die es erlauben die Knoten ohne Suberknoten dynamisch geroutet verwenden zu können. Die Messsensoren können verschiedene Werte messen wie z.B.

\begin{itemize}
	\item Temperatur
	\item Erd- und Luftfeuchtigkeit
	\item Grüne Farbsättigung von Blattwerk
	\item Luftdruck, etc.
\end{itemize}

Diese Daten werden dann in den Systemen nach den Kategorien Operational View und Analytical View unterschieden. Dies dient dazu sie für die entsprechenden DSS- und ERP-Systeme aufzubereiten.

Innerhalb der Anwendungen werden sie nach mathematischen Modelle zur Effizienzsteigerung ausgewertet. Die Ergebnisse werden dann in Form von kurzfristigen Taktiken oder langfristigen Strategien von Expertensystemen und DSS ausgegeben.

Die Strategien und Taktiken betreffen alle Punkte in der Produktionskette wie das Flotten-Management, Resourcenverwendung (Samen, Wasser, Dünger, etc.) oder die Auswahl der geeigneten Anbausorten und Anbauflächen.

\section{Ausblick}

Die Forschung ist vor im Bereich der DSS gefragt und in der Entwicklung von Systemen die zeitnah GIS-Daten erfassen können. Die in \cite{jour:Honkavaara2012} vorgestellte Drohnenlösung bietet ein Forschungsfeld mit viel Potential die teuren und aufwendigen Satellitenbilder ersetzen zu können.

Das in \cite{jour:Dutta2014} vorgestellte Framework stellt einen vielversprechenden Ansatz dar verschiedene Datenquellen auszuwerten um Entscheidungen zu treffen. Dieser Ansatz kann weiter verfolgt werden um weit reichende Verknüpfungen der Umweltdaten für ein Planungssystem zu schaffen das langfristig (wie in \cite{jour:Wenkel2011}) und kurzfristige Abläufe optimieren kann.

Damit kann die zeitnahe Optimierung des Ressourceneinsatz insbesondere für die Landwirtschaft unter freiem Himmel vorangetrieben werden.