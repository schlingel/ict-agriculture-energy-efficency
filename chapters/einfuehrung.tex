\section{Motivation}

Landwirtschaft spielt für jede Gesellschaft eine entscheidende Rolle, da ohne sie die Ernährung der Bürger unmöglich wäre. Für einen Staat ist eine moderne und effiziente Landwirtschaft wichtig um Abhängigkeiten zu anderen Staaten zu verhindern oder zumindest zu verringern. Daher ist dieses Thema auch für Länder der ersten Welt nach wie vor auf der Agenda. Da die Personalkosten hoch sind, ist die Effizenzsteigerung durch Technologie entscheidend für die Entwicklung des Landwirtschaftssektor.

Durch den ständig steigenden Energiebedarf ist vor allem die Frage nach eines optimalen Einsatzes von Energie wichtig für Zukunft der Landwirtschaftsbetriebe in der EU. Neben der Forschung in den Disziplinen der Chemie und des Maschinenbaus, ist die Informatik eine interessante Quelle für kleine und große Optimierungen des landwirtschaftlichen Betriebs. Da ich den Blick über den Tellerrand nicht nur nicht scheue sondern gerne wage und selbst aus einem landwirtschaftlich genutzten Gebiet in Niederösterreich, dem Marchfeld, stamme, liegt mir die Zukunft der Landwirtschaft in Europa am Herzen.

\section{Problemstellung}

Diese Arbeit beschäftigt sich mit der Ausarbeitung des aktuellen Standes der Steigerung der Energieeffizenz in der Landwirtschaft mittels Informations- und Kommunikationstechnik, kurz ICT. Dazu wird eine Zusammenfassung der aktuellen Forschungsprojekte in der EU erstellt und dann eine Zusammenfassung der für Effizenzsteigerung durch ICTs relevanten Literatur erstellt. Dabei wird Wert auf die Ausarbeitung der Forschungsschwerpunkte und Auflistung der für Vergleiche nötigen Kennzahlen.

Ziel ist es eine Übersicht der relevanten Literatur für folgende Arbeiten zu erstellen.

\section{Verwendete Methode}

Die Quellen für diese Arbeit wurden ohne Fokus auf bestimmte Konfernzen oder Datenbanken ausgewählt. Es wurden alle Arbeiten und Projekte die zumindest innerhalb der EU eine Rolle spielen ausgewertet.

\subsection{Literaturrecherche}
Um möglichst keine relevanten Arbeiten zu übersehen wurden neben den akademischen Datenbanken und Bibliotheken auch Berichte von relevanten Forschungsgruppen der EU herangezogen. Die dort erwähnten Projekte und Arbeiten wurden dann gezielt weiter verfolgt. Für die Suche nach Literatur wurden verschiedene Kombinationen aus folgenden Suchbegriffen gewählt:

\textit{energy, efficency, it, informatic, stochastic, agriculture, Landwirtschaft, Effizenz, ICT, Informationstechnologie, Planung}

\subsection{Selektionsvorgang}
Die wissenschaftliche Literatur wurde auf Basis folgender Kriterien bewertet:
\begin{itemize}
  \item ICT-Relevanz. Bei der Suche nach Effizenz in der Landwirtschaft mussten alle Themen aussortiert werden die sich auf Effizenzsteigerung durch chemische Präparate oder bestimmte Entwicklungen im Maschinenbau bezogen.
  \item Veröffentlichungsmedium. Arbeiten die weder im Rahmen einer Konfernz noch in einem Journals oder zumindestens in einem wissenschaftlichen Magazins veröffentlicht wurden, wurden aussortiert.
  \item Aktualität. Die Arbeiten mussten relativ aktuell sein. Werke die vor 2010 geschrieben wurden, wurden nicht weiter verfolgt.
\end{itemize}

Neben wissenschaftlicher Literatur sind auch Reports von aktuellen Forschungsprojekte eine wichtige Quelle. Bei diesen Projekten wurde ebenfalls auf Aktualität geachtet.

\section{Aufbau der Arbeit}

The table of contents is followed by the introduction and the main part, which can vary according to the content. The bachelor's thesis ends with the bibliography (compulsory) and the appendix (optional).

\begin{itemize}
  \item	Cover page
  \item Acknowledgements
  \item Abstract of the thesis in English and German
  \item Table of contents
  \item Introduction (5-10\%)
  	\begin{itemize}
  		\item motivation
  		\item problem statement (which problem should be solved?)
  		\item aim of the work
  		\item methodological approach
  		\item structure of the work
  	\end{itemize}
  \item State of the art / analysis of existing approaches (10-30\%)
  	\begin{itemize}
  		\item literature studies
  		\item analysis
  		\item comparison and summary of existing approaches
  	\end{itemize}
  \item Suggested solution/Implementation/Methodology (20-50\%)
  	\begin{itemize}
  		\item used concepts
  		\item methods and/or models
  		\item languages
  		\item design methods
  		\item data models
  		\item analysis methods
  		\item formalisms
  	\end{itemize}
  \item Critical reflection (10-20\%)
  	\begin{itemize}
  		\item comparison with related work
  		\item discussion of open issues
  	\end{itemize}
  \item Summary and future work (5\%)
  \item Appendix: source code, data models, \dots
  \item Bibliography
\end{itemize}


\section{Recommendations}

\begin{itemize}
	\item Use primary literature, i.e., cite the original work.
	\item Refer diverse source of literature.
	\item Use active language (e.g., 'The programmer prepared a test plan and distributed it to the team' instead of 'A test plan was prepared and distributed to the team by the programmer'). 
	\item Cite all figures and tables in the text (e.g., Fig. 1 shows ...).
	\item Cite figures in the caption.
	\item Redraw cited figures.
	\item Use a spell checker (TexnicCenter => Extras => Options).
	\item Proof read the thesis several times. Let other people proof read your work.
	
	
\end{itemize}



 

