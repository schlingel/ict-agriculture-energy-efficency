\section{Motivation}

Landwirtschaft spielt für jede Gesellschaft eine entscheidende Rolle, da ohne sie die Ernährung der Bürger unmöglich wäre. Für einen Staat ist eine moderne und effiziente Landwirtschaft wichtig um Abhängigkeiten zu anderen Staaten zu verhindern oder zumindest zu verringern. Daher ist dieses Thema auch für Länder der ersten Welt nach wie vor auf der Agenda. Da die Personalkosten hoch sind, ist die Effizenzsteigerung durch Technologie entscheidend für die Entwicklung des Landwirtschaftssektor.

Durch den ständig steigenden Energiebedarf ist vor allem die Frage nach eines optimalen Einsatzes von Energie wichtig für Zukunft der Landwirtschaftsbetriebe in der EU. Neben der Forschung in den Disziplinen der Chemie und des Maschinenbaus, ist die Informatik eine interessante Quelle für kleine und große Optimierungen des landwirtschaftlichen Betriebs. Da ich den Blick über den Tellerrand nicht nur nicht scheue sondern gerne wage und selbst aus einem landwirtschaftlich genutzten Gebiet in Niederösterreich, dem Marchfeld, stamme, liegt mir die Zukunft der Landwirtschaft in Europa am Herzen.

\section{Problemstellung}

Diese Arbeit beschäftigt sich mit der Ausarbeitung des aktuellen Standes der Steigerung der Energieeffizenz in der Landwirtschaft mittels Informations- und Kommunikationstechnik, kurz ICT. Dazu wird eine Zusammenfassung der aktuellen Forschungsprojekte in der EU erstellt und dann eine Zusammenfassung der für Effizenzsteigerung durch ICTs relevanten Literatur erstellt. Dabei wird Wert auf die Ausarbeitung der Forschungsschwerpunkte und Auflistung der für Vergleiche nötigen Kennzahlen. 

Ziel ist es eine Übersicht der relevanten Literatur für folgende Arbeiten zu erstellen.

\section{Verwendete Methode}

Die Quellen für diese Arbeit wurden ohne Fokus auf bestimmte Konfernzen oder Datenbanken ausgewählt. Es wurden alle Arbeiten und Projekte die zumindest innerhalb der EU eine Rolle spielen ausgewertet.

\subsection{Literaturrecherche}
Um möglichst keine relevanten Arbeiten zu übersehen wurden neben den akademischen Datenbanken und Bibliotheken auch Berichte von relevanten Forschungsgruppen der EU herangezogen. Die dort erwähnten Projekte und Arbeiten wurden dann gezielt weiter verfolgt. Für die Suche nach Literatur wurden verschiedene Kombinationen aus folgenden Suchbegriffen gewählt:

\textit{energy, efficency, it, informatic, stochastic, agriculture, Landwirtschaft, Effizenz, ICT, Informationstechnologie, Planung}

\subsection{Selektionsvorgang}
Die wissenschaftliche Literatur wurde auf Basis folgender Kriterien bewertet:
\begin{itemize}
  \item ICT-Relevanz. Bei der Suche nach Effizenz in der Landwirtschaft mussten alle Themen aussortiert werden die sich auf Effizenzsteigerung durch chemische Präparate oder bestimmte Entwicklungen im Maschinenbau bezogen.
  \item Veröffentlichungsmedium. Arbeiten die weder im Rahmen einer Konfernz noch in einem Journals oder zumindestens in einem wissenschaftlichen Magazins veröffentlicht wurden, wurden aussortiert.
  \item Aktualität. Die Arbeiten mussten relativ aktuell sein. Werke die vor 2010 geschrieben wurden, wurden nicht weiter verfolgt.
\end{itemize}

Neben wissenschaftlicher Literatur sind auch Reports von aktuellen Forschungsprojekte eine wichtige Quelle. Bei diesen Projekten wurde ebenfalls auf Aktualität geachtet.

\section{Verwandte Arbeiten}
Das Interesse in eine effiziente Landwirtschaft durch den Einsatz von ICTs wird bereits in \cite{jour:Andreopoulou2012} vorgestellt, auch wenn der Fokus auf Nachhaltigkeit liegt. Effizenz ist dabei nur Mittel zum Zweck. Durch verschiedene Förderungen versucht die EU dies aber voranzutreiben. Eine Übersicht der verschiedenen Forschungsrichtungen wird im Bericht des Projekts \textit{D4.5 Agenda for Transnational Co-operation on energy efficency in agriculture} geboten.\cite{misc:Mikkola2013}. Eine mögliche Stoßrichtung um hohe Effizenz in der Aufzucht von Pflanzen zu erreichen ist \textit{Precision Farming}\cite{jour:Auernhammer2001}.

Die vorgeschlagenen Forschungsschwerpunkte \textit{Sensor technology} wird von den Arbeiten von Zhou Jianjun, Wang Xiaofang, Wang Xiu, Zou Wei und Cai Jichen in \cite{proc:Zhou2013} aufgegriffen. Für Sensoren in Glashäusern haben Mancuso und Bustaffa eine Studie\cite{misc:Mancuso2006} präsentiert die zeigt wie Sensoren Mikroklimas messen und so Pilzerkrankungen verhindern können. Kontextsensitive Landwirtschaftsorganisationssysteme die auf Sensornetzwerken aufbauen werden in \cite{proc:Khaydarov2012} behandelt. Eine Möglichkeit wie diese Daten kostengünstig in einem automatisertem System behandelt werden können, wird in \cite{jour:Kamalesh2014} vorgestellt. In \cite{jour:Shaikh2010} wird ein Framework vorgestellt wie 	kontextsensitive Grid-Systeme gebaut werden können.

Neben \textit{Sensor technology} wird auch die Forschung betreffend \textit{Design Tools} und \textit{Decision Support Systems} angeregt. Ein Vorschlag wie ein solche Planungsprogramme entwickelt werden könnten, wird in \cite{art:Wang2011} vorgeschlagen. Ein Ansatz der GIS-Systeme, Webtechnologie und Data-Mining vereint um ein Expertensystem das verschiedene Bedürfnisse verschiedener Länder beachten kann, wird in \cite{jour:Zhu2009} vorgestellt. Auf die Frage wie solche Daten modelliert werden können, wird in \cite{jour:Schulze2007} behandelt. \cite{jour:Aqeel-ur-Rehman2011} beschäftigt sich mit dem weiterführenden Thema wie Umwelteinflüsse auf den Ertrag modelliert werden können.

Mögliche Hürden die eine Adaption ICT-Lösungen von Landwirtschaftstreibenden und wie diese überwunden werden können, wird in \cite{jour:Aubert2012} vorgestellt.


